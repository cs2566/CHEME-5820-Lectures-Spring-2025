 \documentclass{article}[12pt]
\usepackage{fullpage,graphicx, setspace, latexsym, cite,amsmath,amssymb,xcolor,subfigure}
%\usepackage{epstopdf}
%\DeclareGraphicsExtensions{.pdf,.eps,.png,.jpg,.mps} 
\usepackage{amssymb} %maths
\usepackage{amsmath} %maths
\usepackage{amsthm, comment}

\bibliographystyle{plain}

\newtheorem{theorem}{Theorem}
\newtheorem{prop}{Proposition}
\newtheorem{corollary}{Corollary}
\newtheorem{lemma}{Lemma}
\newtheorem{defn}{Definition}
\newtheorem{ex}{Example}
\usepackage{float}

\newcommand*{\underuparrow}[1]{\underset{\uparrow}{#1}}

\usepackage{amsmath}
\usepackage{graphicx}
\usepackage{xcolor}
\usepackage[dvipsnames]{xcolor}
\usepackage{algorithmicx}
\usepackage{algorithm} %http://ctan.org/pkg/algorithms
\usepackage{algpseudocode} %http://ctan.org/pkg/algorithmicx


\def\R{\mathbb{R}}
\def\Eps{\mathcal{E}}
\def\E{\mathbb{E}}
\def\V{\mathbb{V}}
\def\F{\mathcal{F}}
\def\G{\mathcal{G}}
\def\H{\mathcal{H}}
\def\S{\mathcal{S}}
\def\P{\mathbb{P}}
\def\1{\mathbf{1}}
\def\n{\nappa}
\def\h{\mathbf{w}}
\def\v{\mathbf{v}}
\def\x{\mathbf{x}}
\def\X{\mathcal{X}}
\def\Y{\mathcal{Y}}
\def\eps{\epsilon}
\def\y{\mathbf{y}}
\def\e{\mathbf{e}}
\newcommand{\norm}[1]{\left|\left|#1\right|\right|}
\DeclareMathOperator*{\argmin}{arg\,min}
\DeclareMathOperator*{\argmax}{arg\,max}
\newcommand{\lecture}[4]{
   \pagestyle{myheadings}
   \thispagestyle{plain}
   \newpage
   % \setcounter{lecnum}{#1}
   \setcounter{page}{1}
   \setlength{\headsep}{10mm}
   \noindent
   \begin{center}
   \framebox{
      \vbox{\vspace{2mm}
    \hbox to 6.28in { {\bf CHEME 5820: Machine Learning for Engineers
   \hfill Spring 2025} }
       \vspace{4mm}
       \hbox to 6.28in { {\Large \hfill Lecture #1: #2  \hfill} }
       \vspace{2mm}
       \hbox to 6.28in { {\it Lecturer: #3 \hfill #4} }
      \vspace{2mm}}
   }
   \end{center}
   \markboth{Lecture #1: #2}{Lecture #1: #2}

   \noindent{\bf Disclaimer}: {\it These notes have not been subjected to the
   usual scrutiny reserved for formal publications. }
   \vspace*{4mm}
}


\begin{document}
\lecture{1c}{Unsupervised learning and k-means clustering}{Jeffrey Varner}{}

\section{Introduction}
This lecture introduces the first unsupervised learning approach we will explore: k-means clustering. 
The primary objective of clustering is to divide a dataset into distinct clusters, such that the data points within each cluster exhibit a higher degree of similarity than those in other clusters. 
The k-means algorithm is a straightforward and widely employed method for clustering. The algorithm is easy to understand and implement, and it often produces clusters that are useful in practice. 

The k-means algorithm is an example of an \texttt{unsupervised learning algorithm}. 
Unsupervised learning focuses on discovering patterns and structures in data without the guidance of labeled examples or explicit feedback. 
Unlike supervised learning (which we will explore in future lectures), where algorithms are trained on labeled datasets, unsupervised learning algorithms operate with raw, unlabeled data to identify inherent groupings, anomalies, or relationships. This approach is particularly valuable when dealing with large volumes of unstructured data or when the desired outcomes may be unknown. Typical applications of unsupervised learning include clustering (which we are discussing today), dimensionality reduction, and anomaly detection. Unsupervised learning can provide valuable insights and facilitate data by uncovering hidden structures in data.

\section{K-means clustering}
Fill me in

\bibliography{References-L1c.bib}

\end{document}


