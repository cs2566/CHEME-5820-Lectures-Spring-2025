 \documentclass{article}[12pt]
\usepackage{fullpage,graphicx, setspace, latexsym, cite,amsmath,amssymb,xcolor,subfigure}
%\usepackage{epstopdf}
%\DeclareGraphicsExtensions{.pdf,.eps,.png,.jpg,.mps} 
\usepackage{amssymb} %maths
\usepackage{amsmath} %maths
\usepackage{amsthm, comment}

\bibliographystyle{plain}

\newtheorem{theorem}{Theorem}
\newtheorem{prop}{Proposition}
\newtheorem{corollary}{Corollary}
\newtheorem{lemma}{Lemma}
\newtheorem{defn}{Definition}
\newtheorem{ex}{Example}
\usepackage{float}

\newcommand*{\underuparrow}[1]{\underset{\uparrow}{#1}}

\usepackage{amsmath}
\usepackage{graphicx}
\usepackage{xcolor}
\usepackage[dvipsnames]{xcolor}
\usepackage{algorithmicx}
\usepackage{algorithm} %http://ctan.org/pkg/algorithms
\usepackage{algpseudocode} %http://ctan.org/pkg/algorithmicx


\def\R{\mathbb{R}}
\def\Eps{\mathcal{E}}
\def\E{\mathbb{E}}
\def\V{\mathbb{V}}
\def\F{\mathcal{F}}
\def\G{\mathcal{G}}
\def\H{\mathcal{H}}
\def\S{\mathcal{S}}
\def\P{\mathbb{P}}
\def\1{\mathbf{1}}
\def\n{\nappa}
\def\h{\mathbf{w}}
\def\v{\mathbf{v}}
\def\x{\mathbf{x}}
\def\X{\mathcal{X}}
\def\Y{\mathcal{Y}}
\def\eps{\epsilon}
\def\y{\mathbf{y}}
\def\e{\mathbf{e}}
\newcommand{\norm}[1]{\left|\left|#1\right|\right|}
\DeclareMathOperator*{\argmin}{arg\,min}
\DeclareMathOperator*{\argmax}{arg\,max}
\newcommand{\lecture}[4]{
   \pagestyle{myheadings}
   \thispagestyle{plain}
   \newpage
   % \setcounter{lecnum}{#1}
   \setcounter{page}{1}
   \setlength{\headsep}{10mm}
   \noindent
   \begin{center}
   \framebox{
      \vbox{\vspace{2mm}
    \hbox to 6.28in { {\bf CHEME 5820: Machine Learning for Engineers
   \hfill Spring 2025} }
       \vspace{4mm}
       \hbox to 6.28in { {\Large \hfill Lecture #1: #2  \hfill} }
       \vspace{2mm}
       \hbox to 6.28in { {\it Lecturer: #3 \hfill #4} }
      \vspace{2mm}}
   }
   \end{center}
   \markboth{Lecture #1: #2}{Lecture #1: #2}

   \noindent{\bf Disclaimer}: {\it These notes have not been subjected to the
   usual scrutiny reserved for formal publications. }
   \vspace*{4mm}
}


\begin{document}
\lecture{2a}{Eigendecomposition of Data and Systems}{Jeffrey Varner}{}

\section{Introduction}
The eigendecomposition of a matrix is a fundamental concept in linear algebra. 
In this lecture, we will discuss the eigendecomposition of a matrix, and how it can be used to analyze data and systems 
in the context of unsupervised machine learning.

\subsection*{What is eigendecomposition?}
Eigenvalue-eigenvector problems involve finding a set of scalar values $\left\{\lambda_{1},\dots,\lambda_{m}\right\}$ called 
\href{https://mathworld.wolfram.com/Eigenvalue.html}{eigenvalues} and a set of linearly independent vectors 
$\left\{\mathbf{v}_{1},\dots,\mathbf{v}_{m}\right\}$ called \href{https://mathworld.wolfram.com/Eigenvector.html}{eigenvectors} such that:
\begin{equation*}
\mathbf{A}\mathbf{v}_{j} = \lambda_{j}\mathbf{v}_{j}\qquad{j=1,2,\dots,m}
\end{equation*}
where $\mathbf{A}\in\mathbb{R}^{m\times{m}}$, $\mathbf{v}\in\mathbb{R}^{m\times{1}}$, and $\lambda\in\mathbb{R}$. 
Eigenvalues and eigenvectors are widely used in many areas of mathematics, engineering, and physics:
\begin{itemize}
\item{\textbf{Solution of Linear Differential Equations}: Eigenvectors form a set of linearly independent solutions, while eigenvalues determine the stability of these solutions.}
\item{\textbf{Structural Analysis}: Eigenvalues and eigenvectors describe the structural properties of a matrix or a graph. For example, a structure's natural frequencies and vibration modes, e.g., of a building or a bridge.}
\item{\textbf{Singular Value Decomposition (SVD)}: SVD is commonly used in data analysis, computer vision, image processing, etc, to find the most important features of the dataset.}
\end{itemize}


\section{Computing the Eigendecomposition of a Matrix}
Let $A$ be a square matrix of size $n \times n$. The eigendecomposition of $A$ is given by:
\begin{equation}
A = Q \Lambda Q^{-1}
\end{equation}
where $Q$ is a matrix whose columns are the eigenvectors of $A$, 
and $\Lambda$ is a diagonal matrix whose diagonal elements are the eigenvalues of $A$.

\bibliography{References-L2a.bib}

\end{document}


